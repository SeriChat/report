\chapter{Design}
\label{cha:design}
In this chapter the design of SeriChat which aims to fulfill the requirements defined in chapter \ref{cha:introduction} is presented.

\section{System Overview}
The group chat system, SeriChat, consists of two layers. The first layer is the Kademlia infrastructure based on the TomP2P implementation. The second layer implements the extension with group functionality on top of TomP2P. This layer uses Kademlia functionalities, but it also in some cases communicate independent of Kademlia. 

\section{Group Communication}
Group Communication is created by building a separate tree-structure. The tree-structure is build by help of create, join, and leave commands

Create

Join

Comm

\section{Fault-tolerance}
Fault-tolerance in the Kademlia infrastructure is ensured by TomP2P, but the tree-structure for the group communication is not. The tree-structure is vulnerable because for instance if the root in the tree fails the whole communication goes down. Therefore fault-tolerance for the tree-structure should be handled. There is two scenarios when handling fault-tolerance in the tree-structure. The first is... 
\begin{enumerate}
	\item root
	\item ffs
\end{enumerate}

\section{Confidentiality}
To ensure confidentiality it is important to treat two cases in the group chat. The first case is when a node joins a group the sent password has to be encrypted. As the communication is only between two persons it made most sense to use asymmetric encryption. For this asymmetric encryption the RSA algorithm is used by encrypting the password using the public key for the Group Owner.  
The second case is when a message is sent inside the group, where this message should stay private such that only the group members can read the message. In this case the communication is between one to many and thereby its chosen to use symmetric encryption. For symmetric  encryption the AES algorithm is used. An AES cryptographic key is received by each member joining the group and this key is kept safe using the transfer with help of RSA. All messages between the groups are encrypted by the known AES cryptographic key.  

%Many of the other natural sciences have labs with equipment that has
%to be configured correctly to experimentally test stated hypotheses.
%Such experiments must be planned and designed in advance to work
%properly and provide valid and trustworthy results.
%
%As computer scientists, we usually do not work in labs, and our
%experiments do not live in petri dishes. Still, we have hypotheses to
%test, and thus, experiments to plan. This planning phase is the
%design, where the authors describe the system intended to test the
%hypotheses posed in the introduction.
%
%A luxury of the design chapter is that the design may well go further
%than solely the confirmation or refutation of the hypotheses.  If you
%are building a system, this is where you show that you know how to
%design one, even if you will actually not be implementing all of it.
%If you had sufficient time and resources, this is how you would make
%your system.
%
%However, before we come to that, it is necessary to investigate
%whether the required hypotheses are valid. If they are not, the design
%must be reconsidered, and there is only one way to test them, namely
%through implementation, and subsequent evaluation.
