\chapter{Introduction}
\label{cha:introduction}

This project is intended to build a confidential and fair group chat system based on Kademlia.

%The purpose of the Introduction is make a short (2--6 pages) argument
%that should cover
%\begin{itemize}
%\item What this thesis is about
%\item Why it is interesting or important
%\item What are the central hypotheses that will be investigated 
%\item How will the work be done
%\end{itemize}
%
%This is the place where the reader (who will be a computer scientist,
%but might not be a domain expert) should be convinced that not only is
%the topic interesting and important, the authors have also identified
%central questions/hypotheses pertaining the topic, and have a clear
%plan and methodology to address it.
%
%\section{What makes a good hypothesis?}
%\label{sec:what-makes-good}
%
%For the purposes of a report or thesis, it is wise to concentrate on
%research questions and hypotheses that are quantifiable. \Eg, it is
%better to state that ``method A is better than method B under
%circumstances C'' or ``combining method A with architecture B improves
%on standard approach C'' than ``we can build a system that do X''.
%This is why it is always a good idea to include baselines in your
%work, \ie, established methods or architectural choices that can used
%for comparison. If you do not have baselines yourself, you should at
%least be ready and able to compare your results with the published
%results of others.
%
%The hypotheses should also address central aspects of the work, so
%that \emph{if} these hypotheses are met, the overall work gains in
%credibility, or alternatively (and just as valid), if the hypothesis
%\emph{cannot} be confirmed, it illustrates, why and how the
%assumptions behind the work were flawed, and, hopefully, how they can
%be improved.
%
%\section{Writing a thesis for reading}
%\label{sec:writ-thes-read}
%
%The purpose of the thesis is to be read as a whole, and as such it
%should be written, even if, in reality, it is authored over a period
%of months.  The reader does not naturally understand the flow and
%process of the work involved (this understanding belongs to the
%authors, and upon the authors lies the sole responsibility of
%communicating the work done), and must therefore be guided through the
%work.  In order to accomplish this, the reader should at
%all times have a ready answer in their mind to these questions:
%
%\begin{itemize}
%\item Why am I reading this?
%\item What comes next?
%\item How does this build upon what I just read?
%\end{itemize}
%
%So, why is something there? What is its purpose? How will it used
%later? Vice versa, later in the text, refer back to things established
%earlier (this also supports readers that do not necessarily read
%linearly). While a text grow piecemeal, it is most often read as a
%whole, and should appear as such, lest the reader loses interest.	
%
%To that end, it is a good idea to finish the introduction with a
%description of how the hypotheses are to be investigated, and how this
%is reflected in the structure of the thesis.