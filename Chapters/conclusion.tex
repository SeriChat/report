\chapter{Conclusion}
\label{cha:conclusion}
Building a confidential and fair group chat system based on P2P as a better alternative to the existing solutions had some challenges. 
The most challenging part was to setup and use the TomP2P implementation of Kademlia. TomP2P had some poor and outdated documentation, which made it very difficult to use.  
SeriChat has all important functionality to make a good, fair, and confidential decentralized group chat system  \cite{matl2015effective}. 

%You cannot leave

The results was promising but not real-life situation...




Today we have a lot of different chat systems used for different purposes. These chat systems are build for all kind of operating systems. 
Some of which are not encrypted, others are client-side encrypted and a few are end-to-end encrypted. Common for most of these chat systems are the dependency on centralized solutions which sort of can have some back doors that can create uncertainty about privacy of data. 

Peer-to-Peer (P2P) which is a decentralized way of communicating is an architecture where no one can have a superior role opposed to e.g. a client/server solution or other solutions with third-party interactions. By using P2P one gets rid of the uncertainty for others knowing about data or communication that is meant to be private. 

P2P could be a good alternative for an infrastructure used for chat systems. There already exist designs and implementations of P2P based chat systems e.g. SCRIBE. Though there has always been a challenge in fair distribution of load and this might be the reason why none of these solutions can catch up with today's existing popular client/server based chat systems.  










This, then is the grand summary of what you have accomplished.  You
may well imagine that many readers will read your Introduction, and
then skip to the Conclusion, and if, and only if, those two parts are
interesting, might be tempted to read the rest. A consequence is that
you should ensure that the reader will gain a good overall
understanding of what you have done by reading only the conclusion.
Thus, this is a place to summarise all that has gone before, before
finally concluding on the results of your experiments and the validity
of your hypotheses. It is also important to ensure that the
Introduction (which in all likelihood was written first) still aligns
closely with the conclusions reached.

If you so desire, this is also where you might add a section on Future
Work, where you point in the directions that should be followed to
complete the work you have already accomplished.

