\chapter{Implementation} \label{cha:implementation}
Where the design chapter concerned itself with the overall plan, this
is where the actual experiment in the form of an implementation is
taking form.  It is not the purpose of the implementation to fully
realise the design described in the previous chapter. It is the
exclusive purpose of the implementation (a subset of the design) to
either validate or refute the hypotheses put forth in the
introduction. This, and nothing else. If it does less, you have posed
questions you are not prepared to answer; if it does more, you should
be coding less or asking more questions.

If it illustrates core aspects, \eg, the inner working of a particular
important algorithm or function, code segments are welcome in this
chapter, as long as they are short, to the point, well-commented and
-formatted.  It is also a good idea to provide the reader with a
general overview of the structure of the code, as well as how
communication between various parts take place.  The complete code (as
well as your data) should be included separately with your report in
the form of a zip-file or USB-stick.

Overall, the implementation is the computer scientist's equivalent of
lab equipment carefully arranged into a experimental setup, and just
as the validity of an experimental investigation will be judged in
part on the craftsmanship of the setup, so will the quality of your
implementation. It is therefore important to clearly communicate how
your system works, so that the reader may have confidence in your
evaluation and conclusions.
System 


TomP2P ::/ 

Java -- encryption


not-implemented -- fault-tolerance
