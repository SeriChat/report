\chapter{Related Work}
\label{cha:related-work}
This chapter presents different works done in areas which is related to this project. It includes both some research work and some implemented work. 
The different sections presented in this chapter will be divided into following areas:
\begin{enumerate}
	\item Existing peer-to-peer networks
	\item Existing group communication within peer-to-peer  
	\item Confidentiality and privacy in communication
\end{enumerate}

\section{Peer-to-Peer Networks}
Peer-to-peer (P2P) networking has been evolving for some decades and is now a solid alternative for the existing centralized approach of communicating.
P2P is characterized by sharing resources directly between peers without any intermediate interaction and does not have any single point of failure. The activities are coordinated between the peers in the network.
There are many existing solutions for P2P networks. Int the coming subsections a few of these networks which are relevant for this project  are presented. 
\subsection{Pastry}
Pastry is a structured P2P network which consists of a routing table. 

\subsection{Kademlia}
Kademlia is another structured P2P network functioning as a distributed hash table and which focuses on scalability, loadbalancing and fault-tolerance. Kademila is used in many application with millions of user including BitTorent.

Every Kademlia node owns a 160-bit long unique identifier. The keys is stored at the node with equal id or with shortest distance.
The distance between nodes is determined using bitwise exclusive or (XOR). The XOR is unidirectional and it therefor ensure a unique distance for every node. Unidirectionality means that all lookups for a specific key end at the same path, regardless of the originating node, enabling caching frequently requested key-values and thus lessens the load on nodes with a popular contents.

Nodes in Kademlia is treats as leaves in a binary tree. The position of nodes is determined by the shortest unique prefix of its ID as illustrated by figure x. Every node see the rest of the binary tree as divided subtres. The closest subtree consists of nodes with an id that share the longest common prefix with the node. Every node is ensured to know at least one node in each of the subtres. That info is stored in the k-bucket, which is used for the look-up process. When a node receives a lookup request it will find the node inside its buckets which shares the longest common prefix with the lookup-key. The founded node is then used to another lookup request. This process is repeated until the required node is founded.

Kademila stands out from the other DHT P2P systems by mainting the routing tables with a minimal effort. Every time a node receives a request it will update its k-bucket with info about the requesting node. Thus the k-bucket will stay up-to-date without any manual effort.

%Unlike Pastry, Kademlia uses lookups for discovering a message receiver, and thus a sender first finds the correct message receiver using lookup messages and then directly sends the message to the node. 

%While Kademlia may flood the network with lookups, it does not flood the network with control messages; however, this together may generate more data traffic.

\subsubsection{TomP2P}
TomP2P is a JAVA implementaion of Kademila protocol with many extra features. TomP2P allows for instance automatic set up of NAT traversal, which checks if peers is reachable by its external address. In the case of non reachable node, TomP2P will first try to use UPNP and NATPMP to set up port forwarding on the router and second it will allows setting up distributed relaying. TomP2P enables custom behavior for built-in methods like GET and PUT. Thus developers is able to write some custom code which will be executed when the methods is executed.

%- Replication
%- Security

%Each TomP2P node has a table that can be configured either to be disk-based or memory-based to store its values. 

\section{Group Communication}
\subsection{Scribe}
\subsection{Grup Messaging for Kademlia}

\section{Confidential Communication}
\subsection{Encrypted Communication}
\subsubsection{Asymmetric Encryption}
\subsubsection{Symmetric Encryption}

\subsection{Key Exchange}
\subsubsection{Diffie-Hellman key exchange}
\subsubsection{Symmetric Encryption}








Whereas the purpose of the Introduction chapter was to entice and
convince the reader that work reported is interesting, that the author
is asking the right questions about it, and reading about it will be
worthwhile, the purpose of the Related Work chapter is to
demonstrate that the author possesses a fine overview and keen
understanding of the topic of the work.  Note that while the title of
the chapter is ``Related Work'', it might as well be called
``\emph{Relevant} Work'' in that you should only include work that are
useful or relevant to your purpose. 

Writing about others' work can be challenging---it is easy to succumb
to just writing condensed summaries, which is just as tedious to read
as they are to write. A better method is to gain an overview over the
field of inquiry, and then establish in the first section what aspects
or dimensions are crucial to systems or methodologies such as the ones
described. This demonstrates to the reader that the author has
understanding and judgement. Having done this, every paper or work can
then be described in those established terms. This makes for easier
and much more structured writing, and it also helps the reader
differentiate the systems and works reported on. If there are multiple
works that cover approximately the same area (\eg, using the same
technique), you may mention several, but only go into detail with the
most significant or representative one.

The chapter can then be concluded with a table summarising all the
work reported on using the aspects defined in the introduction of the
chapter.

A crucial element of this chapter is that it concerns the work of
others and \emph{only} that. While the selection of aspects or
dimensions described above invariantly will reflect your own focus,
that should be the extend of which your own work and plans influence
this chapter.  Your own judgement comes in the next chapter.

\section{Frameworks and Technologies}
\label{sec:fram-techn}

Related work need not be only published academic work. In many cases,
it is also relevant to describe crucial frameworks and technologies
that will be used or are relevant for the thesis.  This does not mean
that all employed technologies should be described in detail, but
frameworks and technologies that are unusual (for lack of a better
word) could be described here. \Eg, there is no need to describe an
ordinary network stack, but if the work involves GPU programming, a
description of the chosen architecture might well be relevant, as it
informs all the following chapters.
