\chapter{Related Work}
\label{cha:related-work}
This chapter presents different works done in areas which is related to this project. It includes both some research work and some implemented work. 
The different sections presented in this chapter will be divided into following areas:
\begin{enumerate}
	\item Existing peer-to-peer networks
	\item Existing group communication within peer-to-peer  
	\item Confidentiality and privacy in communication
\end{enumerate}

\section{Peer-to-Peer Networks}
\label{sec:p2p-networks}
Peer-to-Peer (P2P) networking has been evolving for some decades and is now a solid alternative for the existing centralized approach of communicating.
P2P is characterized by sharing resources directly between peers without any intermediate interaction and does not have any single point of failure. The activities are coordinated between the peers in the network.
There are many existing solutions for P2P networks. In the coming subsections a few of these networks, which are relevant for this project, are presented. 
\subsection{Pastry}
Pastry is a structured P2P network designed as a Distributed Hash Table (DHT). DHT is designed to be a infrastructure, which performs lookup services for the rest of the P2P system using keys providing you with values that matches these keys.
The keys and the values are stored in the peers and these can then be used to build knowledge about the peers in the network. Each peer in the network has a routing table which provides some knowledge about the other peers in the network. These routing tables are also maintained by each peer.
The routing table consists of a list of peers with unique id's mapping to the addresses of some peers in the network.
If a peer receives a lookup request it will use the routing table to find a concrete peer also known as node by routing through the nodes with closest id's.

The routing table used in Pastry is based on prefix matching and consists of three layers; Leaf set, Routing table, and Neighborhood set.\\
\begin{figure}[bth]
\includegraphics[width=1\linewidth]{gfx/pastry-routing}
\caption[routingtable]{Routing table for Pastry} \label{fig:pastryrouting}
\end{figure}
When a message is send to a specific node this node's id is looked up in the routing table, where it first will check if the key fits in the range of it's leaf set and if it does, a message can be forwarded to the destination node.
If the key does not fit in it's \emph{leaf set} the routing table inside the pastry routing table is used to forward the message to a node that shares a common prefix with the key by at least one digit.
In rare cases where the appropriate entry is empty or unreachable, the message will be forwarded using neighborhood set to a known node.

%Responsible for the keys closest to the peer..
%Structure knowledge that the individual peer has about the rest of the network
%route 
%use peer you know

\subsection{Kademlia}
\subsubsection{Tom P2P}

\section{Group Communication}
Group communication is an important aspect of communication and can be used to a number of useful things as group chat, live media stream etc.
Regular DHT's or structured networks does not support group communication per default, but there has been build different extensions which does.
\subsection{SCRIBE}
SCRIBE is a very well-known extension build on top of Pastry which supports group communication. 
To create a group in SCRIBE a groupID is generated and a ``create'' message is then routed in the Pastry network to the node whose ID is closest to the groupID. This receiving node becomes the root without knowing it as it only keeps the groupID saved as normal data.
It is now possible to join this newly created group by sending a ``join'' message towards the groupID, then a node in the Pastry network will receive the message and add the sending node as a child and if this node is not a forwarder it will send a ``join'' message towards the groupID and thus become a forwarder for the group. This means there now is a new group member in the group and also a new forwarder. Every node in a group is forwarder, but a forwarder is not necessary a member of  the group.

To leave a group a node will first check if it has any children and if it hasn't is sends a ``leave'' message to its parent which continuous recursively up the tree otherwise it stays as forwarder but not a member.

The tree which is build by SCRIBE is not a part of Pastry, but it is build on top of Pastry. It can be implemented with two different approaches; either as a top-down approache or as crawl.


The choice of implementation for use depends on use case

%Different ways of message passing (Sendign messages):
%1-Top-down --> sequence is easy to handle but huge load on root --> outside of pastry 
%2-Crawl (reminds of Gnutella) --> equal load on all nodes and root but hard to maintain sequence













 
\subsection{Grup Messaging for Kademlia}
Another extension is made a eaearch aetiklce ,,,

\section{Confidential Communication}
\subsection{Encrypted Communication}
\subsubsection{Asymmetric Encryption}
\subsubsection{Symmetric Encryption}

\subsection{Key Exchange}
\subsubsection{Diffie-Hellman key exchange}
\subsubsection{Symmetric Encryption}






