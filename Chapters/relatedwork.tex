\chapter{Related Work}
\label{cha:related-work}
This chapter presents different works done in areas which is related to this project. It includes both some research work and some implemented work. 
The different sections presented in this chapter will be divided into following areas:
\begin{enumerate}
	\item Existing peer-to-peer networks
	\item Existing group communication within peer-to-peer  
	\item Confidentiality and privacy in communication
\end{enumerate}

\section{Peer-to-Peer Networks}
Peer-to-Peer (P2P) networking has been evolving for some decades and is now a solid alternative for the existing centralized approach of communicating.
P2P is characterized by sharing resources directly between peers without any intermediate interaction and does not have any single point of failure. The activities are coordinated between the peers in the network.
There are many existing solutions for P2P networks. In the coming subsections a few of these networks, which are relevant for this project, are presented. 
\subsection{Pastry}
Pastry is a structured P2P network designed as a Distributed Hash Table (DHT). DHT is designed to be a infrastructure, which performs lookup services for the rest of the P2P system using keys providing you with values that matches these keys.
The keys and the values are stored in the peers and these can then be used to build knowledge about the peers in the network. Each peer in the network has a routing table which provides some knowledge about the other peers in the network. These routing tables are also maintained by each peer.
The routing table consists of a list of peers with unique id's mapping to the addresses of some peers in the network.
If a peer receives a lookup request it will use the routing table to find a concrete peer also known as node by routing through the nodes with closest id's.

The routing table used in Pastry is based on prefix matching and consists of three layers; Leaf set, Routing table, and Neighborhood set.\\
\begin{figure}[bth]
\includegraphics[width=1\linewidth]{gfx/pastry-routing}
\caption[routingtable]{Routing table for Pastry} \label{fig:pastryrouting}
\end{figure}
When a message is send to a specific node this node's id is looked up in the routing table, where it first will check if the key fits in the range of it's leaf set and if it does, a message can be forwarded to the destination node.
If the key does not fit in it's leaf set the routing table inside the pastry routing table is used to forward the message to a node that shares a common prefix with the key by at least one digit.
In rare cases where the appropriate entry is empty or unreachable, the message will be forwarded using neighborhood set to a known node.



%Responsible for the keys closest to the peer..
%Structure knowledge that the individual peer has about the rest of the network
%route 
%use peer you know



\subsection{Kademlia}
\subsubsection{Tom P2P}

\section{Group Communication}
Group communication is an important aspect of communication and can be used to a number of things.

\subsection{Scribe}
\subsection{Grup Messaging for Kademlia}

\section{Confidential Communication}
\subsection{Encrypted Communication}
\subsubsection{Asymmetric Encryption}
\subsubsection{Symmetric Encryption}

\subsection{Key Exchange}
\subsubsection{Diffie-Hellman key exchange}
\subsubsection{Symmetric Encryption}








Whereas the purpose of the Introduction chapter was to entice and
convince the reader that work reported is interesting, that the author
is asking the right questions about it, and reading about it will be
worthwhile, the purpose of the Related Work chapter is to
demonstrate that the author possesses a fine overview and keen
understanding of the topic of the work.  Note that while the title of
the chapter is ``Related Work'', it might as well be called
``\emph{Relevant} Work'' in that you should only include work that are
useful or relevant to your purpose. 

Writing about others' work can be challenging---it is easy to succumb
to just writing condensed summaries, which is just as tedious to read
as they are to write. A better method is to gain an overview over the
field of inquiry, and then establish in the first section what aspects
or dimensions are crucial to systems or methodologies such as the ones
described. This demonstrates to the reader that the author has
understanding and judgement. Having done this, every paper or work can
then be described in those established terms. This makes for easier
and much more structured writing, and it also helps the reader
differentiate the systems and works reported on. If there are multiple
works that cover approximately the same area (\eg, using the same
technique), you may mention several, but only go into detail with the
most significant or representative one.

The chapter can then be concluded with a table summarising all the
work reported on using the aspects defined in the introduction of the
chapter.

A crucial element of this chapter is that it concerns the work of
others and \emph{only} that. While the selection of aspects or
dimensions described above invariantly will reflect your own focus,
that should be the extend of which your own work and plans influence
this chapter.  Your own judgement comes in the next chapter.

\section{Frameworks and Technologies}
\label{sec:fram-techn}

Related work need not be only published academic work. In many cases,
it is also relevant to describe crucial frameworks and technologies
that will be used or are relevant for the thesis.  This does not mean
that all employed technologies should be described in detail, but
frameworks and technologies that are unusual (for lack of a better
word) could be described here. \Eg, there is no need to describe an
ordinary network stack, but if the work involves GPU programming, a
description of the chosen architecture might well be relevant, as it
informs all the following chapters.
