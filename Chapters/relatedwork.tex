\chapter{Related Work}
\label{cha:related-work}
This chapter presents different works done in areas which is related to this project. It includes both some research work and some implemented work. 
The different sections presented in this chapter will be divided into following areas:
\begin{enumerate}
	\item Existing peer-to-peer networks
	\item Existing group communication within peer-to-peer  
	\item Confidentiality and privacy in communication
\end{enumerate}

\section{Peer-to-Peer Networks}
\label{sec:p2p-networks}
Peer-to-Peer (P2P) networking has been evolving for some decades and is now a solid alternative for the existing centralized approach of communicating.
P2P is characterized by sharing resources directly between peers without any intermediate interaction and does not have any single point of failure. The activities are coordinated between the peers in the network.
There are many existing solutions for P2P networks. In the coming subsections a few of these networks, which are relevant for this project, are presented. 
\subsection{Pastry}
Pastry is a structured P2P network designed as a Distributed Hash Table (DHT). DHT is designed to be a infrastructure, which performs lookup services for the rest of the P2P system using keys providing you with values that matches these keys.
The keys and the values are stored in the peers and these can then be used to build knowledge about the peers in the network. Each peer in the network has a routing table which provides some knowledge about the other peers in the network. These routing tables are also maintained by each peer.
The routing table consists of a list of peers with unique id's mapping to the addresses of some peers in the network.
If a peer receives a lookup request it will use the routing table to find a concrete peer also known as node by routing through the nodes with closest id's.

The routing table used in Pastry is based on prefix matching and consists of three layers; Leaf set, Routing table, and Neighborhood set.
\begin{figure}[bth]
\includegraphics[width=1\linewidth]{gfx/pastry-routing}
\caption[routingtable]{Routing table for Pastry} \label{fig:pastryrouting}
\end{figure}
When a message is send to a specific node this node's id is looked up in the routing table, where it first will check if the key fits in the range of it's \emph{leaf set} and if it does, a message can be forwarded to the destination node.
If the key does not fit in it's \emph{leaf set} the routing table inside the pastry routing table is used to forward the message to a node that shares a common prefix with the key by at least one digit.
In rare cases where the appropriate entry is empty or unreachable, the message will be forwarded using \emph{neighborhood set} to a known node.

%Responsible for the keys closest to the peer..
%Structure knowledge that the individual peer has about the rest of the network
%route 
%use peer you know

\subsection{Kademlia}
Kademlia is another structured P2P network functioning as a distributed hash table. Kademlia focuses on scalability, loadbalancing, and fault-tolerance. Kademila is used in many application with millions of users including BitTorent.

Every Kademlia node owns a 160-bit long unique identifier. The keys is stored at the node with equal id or with shortest distance.
The distance between nodes is determined using bitwise exclusive or (XOR). The XOR is unidirectional and it therefore ensure a unique distance for every node. Unidirectionality means that all lookups for a specific key end at the same path, regardless of the originating node, enabling caching frequently requested key-values and thus lessens the load on nodes with a popular contents.

Nodes in Kademlia is treated as leaves in a binary tree. The position of nodes is determined by the shortest unique prefix of its ID as illustrated by figure x. Every node see the rest of the binary tree as divided subtrees. The closest subtree consists of nodes with an id that share the longest common prefix with the node. Every node is ensured to know at least one node in each of the subtres. That info is stored in the k-bucket, which is used for the lookup process. When a node receives a lookup request it will find the node inside its buckets which shares the longest common prefix with the lookup-key. The founded node is then used to another lookup request. This process is repeated until the required node is founded.

Kademila stands out from the other DHT P2P systems by mainting the routing tables with a minimal effort. Every time a node receives a request it will update its k-bucket with info about the requesting node. Thus the k-bucket will stay up-to-date without any manual effort.

%Unlike Pastry, Kademlia uses lookups for discovering a message receiver, and thus a sender first finds the correct message receiver using lookup messages and then directly sends the message to the node. 

%While Kademlia may flood the network with lookups, it does not flood the network with control messages; however, this together may generate more data traffic.

\subsubsection{TomP2P}
TomP2P is a JAVA implementaion of Kademila protocol with many extra features. TomP2P allows for instance automatic set up of NAT traversal, which checks if peers is reachable by its external address. In the case of non reachable node, TomP2P will first try to use UPNP and NATPMP to set up port forwarding on the router and second it will allows setting up distributed relaying. TomP2P enables custom behavior for built-in methods like GET and PUT. Thus developers is able to write some custom code which will be executed when the methods is executed.

%- Replication
%- Security

%Each TomP2P node has a table that can be configured either to be disk-based or memory-based to store its values. 

\section{Group Communication}
Group communication is an important aspect of communication and can be used to a number of useful things as group chat, live media stream etc.
Regular DHT's or structured networks does not support group communication per default, but there has been build different extensions which does.
\subsection{SCRIBE}
SCRIBE is a very well-known extension build on top of Pastry which supports group communication. 
To create a group in SCRIBE a groupID is generated and a ``create'' message is then routed in the Pastry network to the node whose ID is closest to the groupID. This receiving node becomes the root without knowing it as it only keeps the groupID saved as normal data.
It is now possible to join this newly created group by sending a ``join'' message towards the groupID, then a node in the Pastry network will receive the message and add the sending node as a child and if this node is not a forwarder it will send a ``join'' message towards the groupID and thus become a forwarder for the group. This means there now is a new group member in the group and also a new forwarder. Every node in a group is forwarder, but a forwarder is not necessary a member of  the group.

To leave a group a node will first check if it has any children and if it hasn't is sends a ``leave'' message to its parent which continuous recursively up the tree otherwise it stays as forwarder but not a member.

The tree which is build by SCRIBE is not a part of Pastry, but it is build on top of Pastry. It can be implemented with two different approaches; either as a top-down approache or as crawl.

%Different ways of message passing (Sendign messages):
%1-Top-down --> sequence is easy to handle but huge load on root --> outside of pastry 
%2-Crawl (reminds of Gnutella) --> equal load on all nodes and root but hard to maintain sequence

The choice of implementation for use depends on use case














 
\subsection{Grup Messaging for Kademlia}
Another extension is made a eaearch aetiklce ,,,

\section{Confidential Communication}
\subsection{Encrypted Communication}
\subsubsection{Asymmetric Encryption}
\subsubsection{Symmetric Encryption}

\subsection{Key Exchange}
\subsubsection{Diffie-Hellman key exchange}
\subsubsection{Symmetric Encryption}






