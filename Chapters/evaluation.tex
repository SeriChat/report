\chapter{Evaluation}
\label{cha:evaluation}

%Having built the equivalent of a experimental setup, it is time to use
%the implementation to test the hypotheses.
%
%This is usually broken down in stages and subquestions.
%
%A structured approach to performing and reporting on experiments is
%to follow this pattern for every single experiment:
%
%\begin{enumerate}
%\item What is the purpose of the experiment?
%\item What is the expected outcome?
%\item What are the parameters under which the experiment takes place?
%\item What are the results?
%\item How do the results align with the expected outcome? If they do
%  not align, why is that so?
%\end{enumerate}
%
%Results should be presented summarised in tables and graphs.  Remember
%to note the number of times experiments were repeated, as well as
%averages, and standard deviations (in percent of the mean).  There is
%much more to the proper evaluation of experimental data than can be
%expounded upon here, but I turn the reader's attention to
%\citep{Downey2011:TSPASFP2011}, which is freely available.